\documentclass{article}

\begin{document}

\title{SYSTEM-PROGRAMMING}
\author{JOHN BWALE}
\date{23-SEPTEMBER-2023}

\maketitle

\section*{Part 1: Data Types and Representation}

\textbf{ASCII Table:}
\begin{enumerate}
    \item Provide a brief explanation of what ASCII stands for.
    
    \textbf{Answer:}
    ASCII stands for American Standard Code for Information Interchange. It is a character encoding standard that assigns unique numeric codes to represent characters in computers and other devices.

    \item Choose three ASCII characters (e.g., 'A', '7', '\$') and list their corresponding decimal, binary, octal, and hexadecimal values.
    
    \textbf{Answer:}
    \begin{itemize}
        \item Character 'A':
            \begin{itemize}
                \item Decimal: 65
                \item Binary: 01000001
                \item Octal: 101
                \item Hexadecimal: 41
            \end{itemize}
        \item Character '7':
            \begin{itemize}
                \item Decimal: 55
                \item Binary: 00110111
                \item Octal: 67
                \item Hexadecimal: 37
            \end{itemize}
        \item Character '\$':
            \begin{itemize}
                \item Decimal: 36
                \item Binary: 00100100
                \item Octal: 44
                \item Hexadecimal: 24
            \end{itemize}
    \end{itemize}

\end{enumerate}

\textbf{Data vs. Instructions:}
\begin{enumerate}
    \item Define what "data" means in the context of computing.
    
    \textbf{Answer:}
    In computing, "data" refers to any information or values that are processed, stored, or transmitted by a computer program. It can include numbers, text, images, audio, or any other type of information that can be represented and manipulated digitally.

    \item Explain what "instructions" are in computing.
    
    \textbf{Answer:}
    In computing, "instructions" refer to the specific commands or operations that a computer program carries out. Instructions are a set of step-by-step actions that tell the computer how to perform tasks, manipulate data, or control the flow of execution within a program.

    \item Provide an example of each: data and instructions.
    
    \textbf{Answer:}
    \begin{itemize}
        \item Example of data: The age of a person (e.g., 25)
        \item Example of instructions: Add two numbers together (e.g., add 5 and 7)
    \end{itemize}

\end{enumerate}

\textbf{Representation of Data:}
\begin{enumerate}
    \item Explain the differences between decimal, binary, octal, and hexadecimal number systems.
    
    \textbf{Answer:}
    \begin{itemize}
        \item Decimal: The decimal number system is a base-10 system that uses ten digits (0-9) to represent numeric values. Each digit's position represents a power of 10.
        \item Binary: The binary number system is a base-2 system that uses two digits (0 and 1) to represent numeric values. Each digit's position represents a power of 2.
        \item Octal: The octal number system is a base-8 system that uses eight digits (0-7) to represent numeric values. Each digit's position represents a power of 8.
        \item Hexadecimal: The hexadecimal number system is a base-16 system that uses sixteen digits (0-9 and A-F) to represent numeric values. Each digit's position represents a power of 16.
    \end{itemize}

    \item Convert the decimal number 75 into its binary, octal, and hexadecimal representations.
    
    \textbf{Answer:}
    \begin{itemize}
        \item Binary: 1001011
        \item Octal: 113
        \item Hexadecimal: 4B
    \end{itemize}

\end{enumerate}

\textbf{2's Complement:}
\begin{enumerate}
    \item Describe what the 2's complement representation is used for in computing.
    
    \textbf{Answer:}
    The 2's complement representation is used for representing both positive and negative integers in computing. It is a way to represent negative numbers by flipping the bits of the corresponding positive number and adding 1 to the result.
    
    \item Explain how to representOops! It seems that the answer got cut off. Here's the continuation of the answer for the second question in Part 1:

```latex
    \item Explain how to represent a negative integer in 2's complement form. Provide an example.
    
    \textbf{Answer:}
    To represent a negative integer in 2's complement form, follow these steps:
    1. Start with the binary representation of the positive value of the integer.
    2. Invert (flip) all the bits.
    3. Add 1 to the resulting value.
    
    Example:
    Let's represent -5 in 2's complement form.
    1. The binary representation of 5 is 00000101.
    2. Inverting the bits gives 11111010.
    3. Adding 1 gives the 2's complement representation: 11111011.

\end{enumerate}

\textbf{Integer Overflow:}
\begin{enumerate}
    \item Define integer overflow.
    
    \textbf{Answer:}
    Integer overflow occurs when the result of an arithmetic operation exceeds the maximum value that can be represented by the data type used for the calculation. In such cases, the result "wraps around" and becomes a value that is smaller or different from the expected result.
    
    \item Give an example where integer overflow can occur, and explain the consequences of it.
    
    \textbf{Answer:}
    Let's consider an example using a signed 8-bit integer. The maximum positive value that can be represented is 127 (01111111 in binary). If we add 1 to 127, the result would be 128, but since it exceeds the maximum value, an integer overflow occurs. In the 8-bit representation, 128 is represented as -128 (10000000 in binary). This can lead to unexpected behavior in calculations and may cause errors or incorrect results in programs that rely on the correct handling of integer values.
    
\end{enumerate}

\section*{Part 2: Software Development Phases}

\textbf{Development, Compilation, Debugging, and Execution:}
\begin{enumerate}
    \item Describe the phases involved in software development, starting from the initial idea to the final product.
    
    \textbf{Answer:}
    The phases involved in software development typically include:
    1. Requirement Analysis: Gathering and analyzing the requirements and objectives of the software.
    2. Design: Creating a detailed plan and structure for the software, including architecture, user interface, and algorithms.
    3. Implementation: Translating the design into actual code using a programming language.
    4. Testing: Evaluating the software's behavior and functionality to ensure it meets the specified requirements.
    5. Deployment: Making the software available for use by end-users.
    6. Maintenance: Monitoring and updating the software to fix bugs, add new features, or address user feedback.
    
    \item Explain the purpose of the compilation phase in software development.
    
    \textbf{Answer:}
    The compilation phase in software development is the process of translating the human-readable source code written by a programmer into machine-readable instructions that can be executed by a computer. The compiler analyzes the code, checks for syntax errors and other issues, and generates an executable file or object code that can be run on a specific platform or operating system.
    
    \item Define debugging in the context of programming and describe its importance.
    
    \textbf{Answer:}
    Debugging is the process of identifying, analyzing, and fixing errors or bugs in a software program. It involves examining the program's behavior, tracing its execution, and using tools and techniques to locate and resolve issues. Debugging is crucial in programming as it helps ensure the correctness and reliability of the software, improves its performance, and enhances the overall user experience.
    
    \item Outline the process of executing a compiled program on a computer.
    
    \textbf{Answer:}
    To execute a compiled program on a computer, follow these general steps:
    1. Load the compiled program into the computer's memory.
    2. Start the execution by running the program's entry point, usually the main function.
    3. The computer's processor interprets and executes the machine instructions in the program sequentially.
    4. The program interacts with the computer's resources, such as input/output devices or memory, as per its instructions.
    5. The program continues its execution until it reaches the end or encounters a termination condition.
    
\end{enumerate}

\section*{Part 3: Standard Streams}

\textbf{Standard Streams (stdin, stdout, stderr):}
\begin{enumerate}
    \item Explain what stdin, stdout, and stderr are in the context of standard streams.
    
    \textbf{Answer:}
    input/output channels that are associated with a running program:
    - stdin (standard input): It is the default input stream from which a program reads data. It is typically used to receive user input or input from another program.
    - stdout (standard output): It is the default output stream to which a program writes its normal output. It is usually used to display results, messages, or data to the user or to redirect the output to a file or another program.
    - stderr (standard error): It is the default output stream for error messages and diagnostic information generated by a program. It is separate from stdout to differentiate between normal output and error output.
    
    \item Describe how stdin, stdout, and stderr are typically used in programming.
    
    \textbf{Answer:}
    In programming, stdin, stdout, and stderr are typically used as follows:
    - stdin: It is used to read input from the user or from another program. Functions like `scanf` in C or `input` in Python can be used to read data from stdin.
    - stdout: It is used to display normal output or results to the user. Functions like `printf` in C or `print` in Python can be used to write data to stdout.
    - stderr: It is used to write error messages, warnings, or diagnostic information. It helps separate error output from normal output, making it easier to identify and handle errors.
    
    \item Explain the concept of redirection of standard streams in a command-line environment.
    
    \textbf{Answer:}
    In a command-line environment, redirection of standard streams allows the user to change the default input or output streams of a command or program. It enables the user to redirect the input from a file instead of the keyboard or redirect the output to a file instead of the screen.
    
    The common redirection operators are:
    - ">" redirects the stdout to a file, overwriting the file if it exists.
    - ">>" redirects the stdout to a file, appending the output to the end of the file.
    - "<" redirects the stdin from a file instead of the keyboard.
    - "2>" redirects the stderr to a file.
    
    Example usage:
    - `command > output.txt` redirects the stdout of `command` to the file `output.txt`.
    - `command < input.txt` redirects the stdin of `command` from the file `input.txt`.
    - `command 2> error.txt` redirects the stderr of `command` to the file `error.txt`.
    
\end{enumerate}

\end{document}