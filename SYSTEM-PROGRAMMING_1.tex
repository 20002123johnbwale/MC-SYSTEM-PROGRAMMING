\documentclass{article}
\begin{document}
\author{JOHN BWALE}
\title{SYSTEM-PROGRAMMING ASSINGMENT (1)}
\date{TODAY}
\maketitle

\section*{System Programming}

\textbf{PART (1). What is a function in programming?}
\begin{enumerate}
  \item[a)] A variable that stores data
  \item[b)] A block of code that performs a specific task
  \item[c)] A mathematical equation
  \item[d)] A data structure
\end{enumerate}

\textbf{In C or C++, which keyword is used to define a function that doesn't return any value?}
\begin{enumerate}
  \item[a)] void
  \item[b)] int
  \item[c)] char
  \item[d)] double
\end{enumerate}

\textbf{What are input arguments in a function?}
\begin{enumerate}
  \item[a)] The values a function returns
  \item[b)] The values passed to a function when it is called
  \item[c)] The values inside a function's code
  \item[d)] The values that are automatically assigned by the compiler
\end{enumerate}

\textbf{What is a return value in a function?}
\begin{enumerate}
  \item[a)] A value used to terminate the function
  \item[b)] The value that a function receives as input
  \item[c)] The value passed to the function when it is called
  \item[d)] The value a function provides as its result
\end{enumerate}

\textbf{What is an array in programming?}
\begin{enumerate}
  \item[a)] A collection of unrelated variables
  \item[b)] A data structure that stores multiple values of the same type
  \item[c)] A function that performs calculations on numbers
  \item[d)] A loop statement
\end{enumerate}

\textbf{In most programming languages, what is the index of the first element in an array?}
\begin{enumerate}
  \item[a)] 0
  \item[b)] 1
  \item[c)] -1
  \item[d)] 10
\end{enumerate}

\textbf{In C and C++, how are strings typically represented?}
\begin{enumerate}
  \item[a)] As arrays of characters
  \item[b)] As single characters
  \item[c)] As integers
\end{enumerate}

\end{document}
